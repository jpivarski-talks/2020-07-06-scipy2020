\pdfminorversion=4
\documentclass[aspectratio=169]{beamer}

\mode<presentation>
{
  \usetheme{default}
  \usecolortheme{default}
  \usefonttheme{default}
  \setbeamertemplate{navigation symbols}{}
  \setbeamertemplate{caption}[numbered]
  \setbeamertemplate{footline}[frame number]  % or "page number"
  \setbeamercolor{frametitle}{fg=white}
  \setbeamercolor{footline}{fg=black}
} 

\usepackage[english]{babel}
\usepackage[utf8x]{inputenc}
\usepackage{tikz}
\usepackage{courier}
\usepackage{array}
\usepackage{bold-extra}
\usepackage{minted}
\usepackage[thicklines]{cancel}
\usepackage{fancyvrb}

\xdefinecolor{dianablue}{rgb}{0.18,0.24,0.31}
\xdefinecolor{darkblue}{rgb}{0.1,0.1,0.7}
\xdefinecolor{darkgreen}{rgb}{0,0.5,0}
\xdefinecolor{darkgrey}{rgb}{0.35,0.35,0.35}
\xdefinecolor{darkorange}{rgb}{0.8,0.5,0}
\xdefinecolor{darkred}{rgb}{0.7,0,0}
\definecolor{darkgreen}{rgb}{0,0.6,0}
\definecolor{mauve}{rgb}{0.58,0,0.82}
\definecolor{titlecolor}{rgb}{0.25,0.34,0.74}

\title[2020-07-06-scipy2020]{\textcolor{titlecolor}{Manipulating JSON-like Data with NumPy-like Idioms}}
\author{\underline{Jim Pivarski}, Ianna Osborne, Peter Elmer}
\institute{Princeton University -- IRIS-HEP}
\date{July 7, 2020}

\usetikzlibrary{shapes.callouts}

\begin{document}

\logo{\pgfputat{\pgfxy(0.11, 7.4)}{\pgfbox[right,base]{\tikz{\filldraw[fill=dianablue, draw=none] (0 cm, 0 cm) rectangle (50 cm, 1 cm);}\mbox{\hspace{-8 cm}\includegraphics[height=1 cm]{princeton-logo-long.png}\hspace{0.1 cm}\raisebox{0.1 cm}{\includegraphics[height=0.8 cm]{iris-hep-logo-long.png}}\hspace{0.1 cm}}}}}

\begin{frame}
  \vspace{1.75 cm}
  \mbox{ } \hfill \includegraphics[width=0.4\linewidth]{logo-600px.png} \hfill \mbox{ }
  
  \vspace{-1 cm}
  \titlepage
\end{frame}

\logo{\pgfputat{\pgfxy(0.11, 7.4)}{\pgfbox[right,base]{\tikz{\filldraw[fill=dianablue, draw=none] (0 cm, 0 cm) rectangle (50 cm, 1 cm);}\mbox{\hspace{-8 cm}\includegraphics[height=1 cm]{princeton-logo.png}\hspace{0.1 cm}\raisebox{0.1 cm}{\includegraphics[height=0.8 cm]{iris-hep-logo.png}}\hspace{0.1 cm}}}}}

% Uncomment these lines for an automatically generated outline.
%\begin{frame}{Outline}
%  \tableofcontents
%\end{frame}

% START START START START START START START START START START START START START

\begin{frame}[fragile]{What does {\it that} mean?}
\vspace{0.3 cm}

\small
\begin{minted}{python}
>>> import awkward1 as ak
>>> import numpy as np
\end{minted}

\begin{minted}{python}
dataset = [
    [{"x": 1,  "y": [101]},
     {"x": 4,  "y": [102, 202]},
     {"x": 9,  "y": [103, 203, 303]}],
    [],
    [{"x": 16, "y": [104, 204, 304, 404]},
     {"x": 25, "y": [105, 205, 305, 405, 505]}]
]
>>> array = ak.Array(dataset)
\end{minted}

\begin{uncoverenv}<2->
\begin{minted}{python}
>>> array[-1, :, "y", 2]
<Array [304, 305] type='2 * int64'>
\end{minted}
\end{uncoverenv}

\begin{uncoverenv}<3->
\begin{minted}{python}
>>> array[:, "y", :, 0] - np.sqrt(array[:, "x"])
<Array [[100, 100, 100], [], [100, 100]] type='3 * var * float64'>
\end{minted}
\end{uncoverenv}
\end{frame}

\begin{frame}{Table of contents}
\vspace{0.5 cm}

\large
\begin{description}
\item[1:00\hspace{0.5 cm}] \textcolor{gray}{(This table of contents)}
\item[2:00\hspace{0.5 cm}] Why does this library exist? Motivation from particle physics
\item[5:00\hspace{0.5 cm}] ``Live'' demo exploring a dataset: Chicago bike paths
\item[10:00\hspace{0.5 cm}] Data types and generalizing NumPy
\item[13:00\hspace{0.5 cm}] Using Awkward Arrays in Numba
\item[15:00\hspace{0.5 cm}] Building complex structures (still in Numba)
\item[17:00\hspace{0.5 cm}] Interoperability with Pandas, Arrow, NumExpr\ldots
\item[18:00\hspace{0.5 cm}] How it works: columnar transformations
\item[20:00\hspace{0.5 cm}] Software architecture from Awkward 0.x to 1.x
\item[23:00\hspace{0.5 cm}] Development of a GPU backend
\item[25:00\hspace{0.5 cm}] Conclusions
\end{description}
\end{frame}

\begin{frame}{}

\end{frame}

\end{document}
